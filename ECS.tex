\documentclass{article}
\usepackage{url}
\usepackage[english]{babel}
\usepackage{graphicx}

\title{Implementation and usage of \\
        Entity Component Systems}
\author{Benedikt Danner}
\date{SJ2021-2022}
\bibliographystyle{plain}

\begin{document}
    \maketitle
    \begin{abstract}
        
    \end{abstract}
    This article aims to explain the usages and the design challanges of an \textit{Entity Component System} or \textit{ECS} for short. Different implementations of the indiviual parts of any ECS are considered and their advantages and disadvantages outlined.
    \pagebreak

    \tableofcontents    
    \pagebreak

    \section{The initial problem}
    \subsection{Software Architectural Patterns}
    During the development of any software project certain criteria have to be met. While there are certain functional requirements their are most certainly also certain technical ones. These technical requirements can range from memory usage and runtime speeds to code archticture requirements, such as code maintainability and readabilty, test coverage, etc\dots.
    
    Luckily most project have many things in common and a solution that works for one project might also work in another one. That is why certain patterns have emerged in the world of software architcture to solve some common problems and the \textit{Entity Component System} is one such pattern. It is used primarily in the game industrie for game engine development, but has found many other applications in areas where a data driven architcture is used.

    \subsection{Shared Properties}
    When developing in a data driven approach one problem tends to occur very quickly. Shared propteries. One property occurs in a wide number of different objects, but they all represent the same thing and all objects with that property should thus be handled at the same location in order to reduce code redundancy.

    \subsection{Application in the game industry}
    For clearification purposes, consider a simple RPG game world with a player character, a monster, and a table in it. The player and the monster hit each other as they walk around the table. One thing all three entities have in commain is the collision box. Otherwise it would be possible to walk through the table and impossible to hit the monster or the player. However only player and the monster have health bars since the table should be indestructble. 
    This is a perfect example for shared properties.

    \subsection{Application in graphical user interfaces}
    Another use case for and ECS in the implementations of graphical user interfaces. Both buttons and text fields can have borders. As such they should be handled at the same location. Having to define a border for each item would be every software engineers nightmare.

    \subsection{ECS is a tool}
    The last example is deliberaly made a little bit of a stretch. Most GUI frameworks do not use a ECS in the backend and for good reason. ECS is only a tool with advantages and disadvantages. Even if the usecase might sound like a good fit for a tool, there might be a even better one, depending on the metric you choose to go by. Games often need to be high performant because they need to handle physics, graphics, gameplay and AI at 60 FPS, while GUIs can be a little bit more lenient with the optimizations. 
    Good ECS systems are designed to reduce memory usage, lookup speeds and cache misses as much as possible in order meet the technical requirements of a game. GUIs on the other hand can afford to trade a little bit of performance for simplicity.

    \section{The inheritance model}

    \section{The \textit{Entity Component System}}
    \subsection{Pure \textit{Entity Component Systems}}
    \subsection{The Component}
    \subsection{The Entity}
    \subsection{The System}
    \subsection{The Registry}

    \section{Conclusion}

    \bibliography{ECS}
\end{document}